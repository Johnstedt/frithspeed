\chapter{Departure}

\vspace{-1.3cm}
\begin{localsize}{10}
	\begin{quote}
	%	"Végre nem butulok tovább"\\
		
	%	(Finally I am becoming stupider no more) 
	%	\begin{flushright}- Paul Erdős \end{flushright}
	
	“Hell is truth seen too late.”
	\begin{flushright}- Thomas Adam \end{flushright}
	\end{quote} 
\end{localsize}
\vspace{1cm}

There is no honorable discharge for the weak and injured in a war of nutrition

The winter was mild, yet the animals died like fly larvae in drying puddles. One could believe the forest was interlocked in a civil war with many causalities on both sides. The war however, was one against nutrition and cold. \\

Starvation is worse than war. In war it is the able, the strong and the fierce, that die. They oft do so in a noble cause - if the cause is just and victorious history may even smile at them, etching their names into vast stone memorials. 

Starvation is another beast entirely, it's the weak that fall first - the kittens, fawns and the newly hatched and spawned - the elders and wise, the crocked and hurt. The loving old wise doe after a long fulfilled life; dies but in agony without so much as dignity left in her soul. The starved follow Dylan Thomas's advice, they truly 'Do not go gentle into that good night'. They go in extravagant desperation, searching, raging, for the light. Howling in pain for all the burrow to hear. In the beginning they may receive empathy for their pains, but when they die they've long drained their kin dry and only spite remains.\\

Kraerion, at last his fathers son, had taken his responsibilities to heart. He had lead the chipmunks daily work for almost a year. They had worked hard he thought to himself. "Not hard enough", his own thoughts hurt him more than any conceivable torture ever could. 

(Initiative)

He narrow run widened under the support of the roots of the old oak. He stopped, he heard laughter in the distance - but that was not why he stopped. The hollows between the roots were filled with large heaps of food. Piles of mushrooms, berries and nuts. There was even dried berries, conserved and safe from rot - which he had never seen before. He moved between the heaps, still grasping the extravagant wealth - fighting the dissonance in his mind. "Surely my men must have passed here many times", the thought to himself not . He passed the heap with the screw-nuts. It looked like a mountain next the Chipmunks now meager body. Next to the heap stood a stack of fresh green leaves, preserved from the autums fire.

He heard laughter again. He moved passed into what may only be described as a room. The damp walls and floor were covered with beautiful flat stones, laid meticulously. Large wooden beams held up the ceiling and through a few holes; light slippered into the room. In the middle of the room a large wooden table full of carved details. All around the table sat crows feasting and laughing. Along the wall hang, what a human might call art - but to Kraerion it didn't look like anything. At the edge of the table, Eaglewing sat. He looked up and saw Kraerion standing in the entrance.\\

"Craax", he cried pointing his brown wing in the air. "I wondered when you would join us.", when Kraerion did not reply he continued, ""


"Did you really believe we wouldn't wield the power given to us,"

Kraerion finally understood. The hunger he had suppressed for so long roared at him from within. He stumbled, as if he was intoxicated, and the crows roared with laughter. The realization utterly broken him. He sought the path away from the room, but he could barely manage to stand on his feet.

He had remained strong when his brother died. He had remained strong when his litter starved to death. When Cinnamon lost her laughter, he had carried on. He had sacrificed his dreams for duty and honor. It had all been for naught. The long suppressed grief came to him at once, carving him to the bones.

Sunlight appeared before him, the end of the run. He tumbled outside into snow, into the cold white landscape. He weakened and laid down. As he stared up at the sun for the last time he thought solemnly.

"What is strength? I don't know anymore. I did my duty, I lived what was right and I worked harder than anyone. Yet all I love is dead."

He knew that he was a fool, but he did not understand why or how. He also knew that no chipmunk, no rabbit nor any roe could grasp the depths of the scheme of the murderous crows nor have the ability to halt it.

He understood now. He understood that he did not understand. He knew he was a fool; although he did not grasp the depths of the scheme the crows had created; he knew that he had swallowed it whole. 
He knew that his, and his fathers, diligent character and the trust in it had been the facade the crows had hid behind.

"What ought to be right and true could not have been. I thought working was right, when making right was there all along."

The cold caught his heart still. He had had his last thought. 

He had remained oblivious during the discourse, knowing he didn't understood and accepted it. He had not made it his duty to understand, to fend for his interests. He had believed their intentions, yet been blind to their cause. He had naively toiled, sacrificed himself for all he held dear, for all animal kind. 

Duty is but a reflection of power. It's both war and peace; neither virtue, nor vice. It's simply moral apathy. A decision prescribed, stamped and approved by some proclaimed vassal of God. Duty is the moral equivalence to serfdom. And that is the road Kraerion had traveled, heeding the advice of Pietas, down the road to serfdom. And whilst admirrable in good times, a collective neck by which to tie the not knot in bad. 

The last captain dead, the ship weather'd true enough but only sought in vain. The bleeding drops of red are redder still with no objective won. What is bravely spilled may just be spilled, it seldom carry purpose. For you the shores, the crowded shores, are born into serfdom. The decree of equity only but a paper - a shadow tax is still a tax unrepresented.

The stark ways he had lead his life - had been nothing but folly. He had been proud for what he represented, he had worked hard and done what ought to be right. 


Death is not the opposite of life as ignorance is not the opposite of knowledge — only a lack thereof. Life is a cell dividing — growth itself; its opposite must thus be the opposite motion: live and become less. 'A virus!', you may be inclined to proclaim, yet the virus only transforms life, as the fox snatches a rabbit — a life for another. 
a force of perpetual serfdom
Whatever the opposite of life is, we know this: Kraerion died, yet he did not lose his life — he had no life left to lose.

Wherever the opposite of life may lay, we know this to be true: Kraerion died, yet he did not lose his life; he had no life left to lose.

The stark difference of life and death may make them appear as opposites, but they are not. Death is not the opposite of life, as ignorance is not the opposite of knowledge — only a lack thereof. Absence may highlight the worth of the absentie, as a sudden death reveals the life that is lost.
A life not lived, to . The proverbial marrow left unsucked. Live and Lived are also palindromes — another form of opposition — and maybe they tell the final truth.
Whatever the opposite of life may be, we know this to be true: Kraerion did die, but he did not lose his life; he had no life left to lose.


Theirs not to make reply,
Theirs not to reason why,
Theirs but to do \& die


The proverbial life's marrow left unsucked