\chapter{Delight}

\vspace{-1.3cm}
\begin{localsize}{10}
	\begin{quote}
		“If a man does not keep pace with his companions, perhaps it is because he hears a different drummer. Let him step to the music he hears, however measured or far away.”
		\begin{flushright}- Henry David Thoreau \end{flushright}
	\end{quote} 
\end{localsize}
\vspace{1cm}

The sun slowly rose in the eastern sky; innocent rays of morning trickled through the dense forest canopy. The sparse beams reaching for the undergrowth, fighting off the morning mist like a potent antidote. The daily struggle between dew and vapor began anew. Kraerion sat awake in his hollow, hunched on his forepaws, gazing contently out through the morning air. He had woke---like most mornings---to the varied orchestra of nightingales and blackbirds, tirelessly competing for the vibrations of the humid air. The beatuiful song was now and then disrputed by the distinct cooing of one of the neighbouring wood pigeons.

Kraerion had managed well: gathering his own feed, sumplimented by the odd job when such oppertunities arose; but---of course---he was less efficient now then when he was gainfully employed. Yet, he enjoyed his new savage life, what he lost in produce, he gained in beauty. The absence of the natural irregularities of life---which a steady piece of produce, from an even steadier wage, had smoothen numb---was to him by itself: disutility. And, waking before the sun instead of with it---there's disutility there too.

Before he left his burrow and former life, he had saved for a long time. He had manage to obtain his small fortune of seven screw nuts by sheer hard work; three for the birch, one for Redrill, and the three remaining to act as a safety net during the transition to independence, and cover any unforseen  uncertainty that his new life surely could entail.

But, that summer, he was blessed with something magical: prices began to fall. First little by little, and then all at once. By the time the sun reached it summer solstice, the prices had dropped to about half of what they had been during the spring---making his savings not only ample for his transition, but ample for anything the future could possibly bring.

No cause could be attributed to the sudden drop in prices, but thought or concern did not burden the Wadborough animals' minds; instead, the feasted as if the abundance of summer would never depleat. % break

The exchange to leaves had happened as the Eaglewing dictated exactly one week after the meeting in the glade. The animals had formed a queue before the Old Oak, to attempt an orderly, pedestrian proceeding. Although the animals had increased in affluence, queues are still as innate to them as a sheep stalking a pack of wolves, or a chicken snagging a fox for dinner; thus, it was short of a miracle that no major incident occured. Kraerion---ever so careful---had waited til the absoult end to avoid most of the expected tumult before exchanging his remaining screw nuts for a thin stack of leaves.

A sound of scratched bark broke his thoughts. As so often happened to Kraerion, his thoughts remaind but half processed---flung back to reality with a mind in an unfit state. He leaned his head out of his hallow to see down the trunk and saw a familiar shape standing at the foot of the birch.

"I'm not climbing this \textit{thing}!", Cinnamon cried mockingly. Kraerion moved, in response, down the tree and stopped a rought meter from the ground, in a manner not dissimilar to that of a nuthatch.

"You never said goodbye.", she said at length when Kraerion failed to speak.

"What could I have said?", he answered, knowing full well how stupid it sounded, "Besides, I don't think I could---it was difficult enought as it was."

"Well, if it was too difficult---might it not be the wrong decision?"

"No!", he cried harhsly as by reflex, "It was truly the best decision I've ever made."
As usual, he spoke before thinking and missed to consider Cinnamon's feelings---now she looked rather hurt. 

"I miss you.", he added quickly, "You know what I mean. You know how miserable I was, I wouldn't have been good for you or for anyone if I stayed."

"I guess you're right.", she whispererd, mulling it over.

"Do you want to walk somewhere?", he asked, more to get out of the awkward silence than anything else.

"Somewhere?"

"Anywhere."

"And here I thought you were the one to distinct between walk and wander; but sure, we can walk." she remarked, making something that could be construed as a smile, before turning. She began walking eastward, before he had a chance to reply and he was left little choice but to follow and catch up with her.

Kraerion had long made an effort to distinguish between walking and wandering. "Walking is active and with purpose---you \textit{walk} with the purpose to get somewhere.", he would preach, "But wandering \textit{is} the purpose, done for its own sake---and end in itself." To Kraerion they were more than just definitions, they symbolised the two different outlooks of life: the former, the ambitious numbness of habit; the latter, the very beauty of life itself. He reasoned that ones quality of life could be deduced from this innocent choice of words. That someone who walk instead of wander, would eventually met with the Thoreauvian fear: to wake up one day and discover they had not lived. The chipmunks---and thus also Cinnamon---had built their existance around transportation, schedules, and habit; so, whilst she'd always listen, he feared she'd never truly hear.

After wandering for some time, Cinnamon broke the silence. 

"I'm worried about your sister,", she began tentatively, "she's changed."


\begingroup
	\fontsize{10pt}{12pt}\selectfont
		\begin{quote}
			Our sun be eternal bound,
			Alone she goes around;
			Her shackles that of natural law,%Her loneliness that of natural law —
			Yet she smiles; and thy bones thaw.

			And trees the east salute,
			To whom that bears them fruit.
			Although raised in different glades,
			They cast but similar shades.

			What an angle may disclose,
			To thee, and he who knows; % SPELLING ERR To thee, and he who knows, and those who knows
			With stick and stones to be precise,
			Of where a shadow bound to slice.

			A sun's eternally bound,
			To us distant yet profound —
			To tell the pace of the Sublime,
			And thus thou know about the Time.
		\end{quote} 
\endgroup


%A sun's eternally bound,
%Companionless, but she must around;
%Her solitude that of natural law,%Her loneliness that of natural law —
%Yet she smiles; and thy heart thaw.

%And trees the east salute,
%To whom that bears them fruit.
%Although raised in different glades,
%They cast but similar shades.

%What an angle may disclose,
%To thee, and he who knows; % SPELLING ERR To thee, and he who knows, and those who knows
%With stick and stones to be precise,
%Of where a shadow bound to slice.

%A sun's eternally bound,
%To us distant yet profound —
%To tell the pace of the Sublime,
%And thus thou know about the Time.


