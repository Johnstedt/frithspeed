\chapter{Delight}

\vspace{-1.3cm}
\begin{localsize}{10}
	\begin{quote}
		“If a man does not keep pace with his companions, perhaps it is because he hears a different drummer. Let him step to the music he hears, however measured or far away.”
		\begin{flushright}- Henry David Thoreau \end{flushright}
	\end{quote} 
\end{localsize}
\vspace{1cm}

The sun slowly rose in the eastern sky; innocent rays of morning trickled through the dense forest canopy. The sparse beams reaching for the undergrowth, fighting off the morning mist like a potent antidote. The daily struggle between dew and vapor began anew. Kraerion sat awake in his hollow, hunched on his forepaws, gazing contently out through the morning air. He had woke---like most mornings---to the varied orchestra of nightingales and blackbirds, tirelessly competing for the vibrations of the humid air. The beautiful song was now and then disrupted by the distinct cooing of one of the neighbouring wood pigeons.

Kraerion had managed well: gathering his own feed, supplemented by the odd job here and there when such opportunities arose; but---of course---he was less efficient now then when he was gainfully employed. Yet, he enjoyed his new savage life, what he lost in produce, he gained in beauty. The absence of the natural irregularities of life---which a steady piece of produce, from an even steadier wage, had smoothed numb---was to him by itself: disutility. And, waking before the sun instead of with it---there's disutility there too.

\renewcommand*{\thepage}{\footnotesize \hexadecimal{page}}

Before he left his burrow and former life, he had saved for a long time. He had manage to obtain his small fortune of seven screw nuts by sheer hard work; three for the birch, one for Redrill, and the three remaining to act as a safety net during the transition to independence, and to cover any unforeseen uncertainty that his new life surely could entail.

But, that summer, he was blessed with something magical: prices began to fall. First little by little, and then all at once. By the time the sun reached its summer solstice, the prices had dropped to about half of what they had been during the spring---making his savings not only ample for his transition, but ample for anything the future could possibly bring.

No cause could be attributed to the sudden drop in prices, but thought or concern did not burden the Wadborough animals; instead, they feasted as if the abundance of summer would never deplete.% break

The exchange to leaves had happened as the Eaglewing had dictated exactly one week after the meeting in the glade. The animals had formed a queue before the Old Oak, to attempt an orderly, pedestrian proceeding. Although the animals had increased in affluence, queues are still as innate to them as a sheep stalking a pack of wolves, or a chicken snagging a fox for dinner; thus, it was short of a miracle that no major incident occurred. Kraerion---ever so careful---had waited until the absolute end to avoid most of the expected tumult before exchanging his remaining screw nuts for a thin stack of leaves.\\

A sound of scratched bark broke his thoughts. As so often happened to Kraerion, his thoughts remained but half processed--- he was flung back to reality with a mind in an unfit state. He leaned his head out of his hallow to see down the trunk and saw a familiar shape standing at the foot of his birch.

\renewcommand*{\thepage}{\footnotesize \arabic{page}}

"I'm not climbing this \textit{thing}!", Cinnamon cried mockingly. Kraerion moved---in response---down the tree and stopped a rought meter from the ground, in a manner not dissimilar to that of a nuthatch.

"You never said goodbye.", she said at length when Kraerion failed to speak.

"What could I have said?", he answered, knowing full well how stupid it sounded, "Besides, I don't think I could---it was difficult enough as it was."

"Well, if it was too difficult---might it not be the wrong decision?"

"No!", he cried harshly as by reflex, "It was truly the best decision I've ever made."
As usual, he spoke before thinking and missed to consider the product of his words---now Cinnamon looked rather hurt. 

"I miss you.", he added quickly, "You know what I mean. You know how miserable I was; I wouldn't have been good for you or for anyone if I had stayed."

"I guess you're right.", she whispered, mulling it over.

"Do you want to walk somewhere?", he asked, more to get out of the awkward silence than anything else.

"Somewhere?"

"Anywhere."

"And here I thought you were the one to distinct between walk and wander; but sure, we can walk." she remarked, making something that could be construed as a smile, before turning. She began walking eastward, before he had a chance to reply and he was thus left with no choice but to follow and catch up with her.\\

Kraerion had long made an effort to distinguish between walking and wandering. "Walking is active and with purpose---you \textit{walk} with the purpose to get somewhere.", he would preach, "But wandering \textit{is} the purpose, done for its own sake---an end in itself." To Kraerion they were more than just definitions, they symbolized the two different outlooks of life: the former, the ambitious numbness of habit; the latter, the very beauty of life itself. He reasoned that ones quality of life could be deduced from this innocent choice of words. That someone who walk instead of wander, would eventually met with the Thoreauvian fear: to wake up one day and discover they had not lived. The chipmunks---and thus also Cinnamon---had built their existence around transportation, schedules, and habit; so, whilst she would always listen, he feared she would never truly hear.\\

After wandering for some time, Cinnamon broke the silence. 

"I'm worried about your sister,", she began tentatively, "she's changed."

"Veritas?", Kraerion said.

"I... --- I don't know. Well, maybe not changed per se. She's herself alright, only more so."

"Honesty, can you too much of that?", Kraerion chuckled, "I think I see what you mean though. She can get stuck in her mind sometimes, not letting thoughts go---like caught in a never ending loop. But she's sharp, I'm sure she'll figure it out."

"I don't know, maybe...", she said slowly, "Maybe that's why we call someone smart sharp, they cut wherever they turn---themselves too if not supervised. Whilst, someone dull couldn't hurt a fly if they tried. But you know what Kraerion, you were always her whetstone; always there to keep her on track. And well, you're not there anymore..."

"Nonsense, you talk as if she's crazy!"

"Maybe she is! She's at least beginning to be perceived as if she is. But, there's more, she's started talking about our economy as if there's some great danger to it. She's not terribly coherent in the best of times, but now she's regressed to repeating...---repeating these incomprehensible phrases."

"What do you mean phrases---what phrases?"

"There's a bunch of them, but there's a couple in particular that she keeps repeating which I managed to remember: 'feed is dialectic; hunger is constant', and 'The delta must eventually reverse'", she paused, "Do they tell you anything?"

"Well...", he thought for a while, "The first is clearly about winter. Feed is cyclical: from the first rare morels that nigh beat thaw, to the late autumn tubers that's accessible until the ground freezes. And, hunger is..., well hunger: constant and independent of season. But why would she bring it up now? That's always been the case, and the Winter Fund was invented to fix it. She's \textit{practically} running the fund now, in all but name.", he took another pause, "the other one I don't understand though."

Their conversation did not end, but transposed to one in Silence. With words omitted, they traversed through the forest. Animals communicates more with their bodies than humans, and the two chipmunks soon fell into the comforts they so many times before had felt in each others presence. Nevertheless, there was much left unsaid between them and a tension slowly grew. Cinnamon had not shown up but for a sense of duty: to tell of Veritas well-being; both new it, yet they left it unresolved.

"You are never coming back, are you Kraerion?", she said when she finally had had enough of the silence, stopping abruptly, her two chipmunk eyes---glimmering, like black pearls caught in sunlight---glaring straight through him.

"No.", he said staunchly after some thought, "I'm sorry, but I don't think I ever will."

"But why---why would you leave? I'm not asking for Veritas, or Aequitus, or anyone else; I'm asking for me, Kraerion. For me. What about us---what about the life we were creating?"

"I don't know what more I can say...", he said sadly, "I can explain how awful I find life in darkness with that damnable filthy soil all around, how it's like my body completely rejects it; as you seem to reject the notion to live in the trees. No matter how much I want to be with you, I can never be in consonance with a life underground. I assure you, it hurts me as much I think it hurts you, to not see you everyday."

Once again, they regressed to Silence. Exhausted by their imposition, they wandered along.

They wandered past the rabbits' great hill, thence past the clearings which constitute the central parts of the forest. Many an animal laid leisurely basking in the morning sun, and they watched as the two chipmunks quietly wandered past. "That's queer,", Kraerion thought, "aren't they suppose to be working at this hour?", but he soon dismissed the thought.\\

When they reached the Glade of Representatives. Kraerion could scarcely recognize the glade without it being filled by the forest's animals. The glade however, wasn't empty; on the contrary, woodpeckers were swarming the Old Oak---pecking away at its trunk; sturdy roebucks were hauling loads of dirt away from the glade; and, crows were all around, talking amongst themselves---making plans. 

They had nigh entered the glade before a voice called out for them. 

"Good morning!", the voice cried, which they now saw belonged to Hroðulf, as he waved them to come over. He was standing next to a straight stick standing perpendicular to the ground; stones had been laid out around the stick to form a semi-circle. In between every stone, lay a couple of couple of pebbles.

"Morning,", Cinnamon responded as they approached, "What is this?", she continued as she pointed at the stick and stones. Hroðulf only smiled for a while, then he spoke:

\begingroup
	\fontsize{10pt}{12pt}\selectfont
		\begin{quote}
			Our sun be eternal bound,
			Alone she goes around;
			Her shackles that of natural law,%Her loneliness that of natural law —
			Yet she smiles; and thy bones thaw.

			And trees the east salute,
			To whom that bears them fruit.
			Although raised in different glades,
			They cast but similar shades.

			What an angle may disclose,
			To thee, and he who knows; % SPELLING ERR To thee, and he who knows, and those who knows
			With stick and stones to be precise,
			Of where a shadow bound to slice.

			A sun's eternally bound,
			To us distant yet profound —
			To tell the pace of the Sublime,
			And thus thou know about the Time.
		\end{quote} 
\endgroup


%A sun's eternally bound,
%Companionless, but she must around;
%Her solitude that of natural law,%Her loneliness that of natural law —
%Yet she smiles; and thy heart thaw.

%And trees the east salute,
%To whom that bears them fruit.
%Although raised in different glades,
%They cast but similar shades.

%What an angle may disclose,
%To thee, and he who knows; % SPELLING ERR To thee, and he who knows, and those who knows
%With stick and stones to be precise,
%Of where a shadow bound to slice.

%A sun's eternally bound,
%To us distant yet profound —
%To tell the pace of the Sublime,
%And thus thou know about the Time.

Just when he stopped talking---as by a bespoke request to God---the sun appeared behind the lone cloud in the sky. Its rays of sunlight gave birth to a shadow running from the stick to the second westernmost stone.

"Wait---", Cinnamon said astonished, "do you mean that this thing---this \textit{contraption}---tell us the time?"

"Yes!", the crow hooted with delight, "And here I thought \textit{rodents} were void of wit! Whilst the sun travels above us from east to west, its shadows travels from west to east. The stones simply enumerates the Time of day."

"And this shadow, is it precise?", Cinnamon asked, ignoring the slight, "Is it at the same stone at the same time every day?"

"Not only is it precise, it's also completely independent of location. No matter where in the forest, the angle of the shadow is exactly the same. We can just raise the same structure in all parts of our Forest, which I presume you chipmunks---with your vocation---see the value in."

"This is truly \textit{brilliant}, nothing will be the same!"

"No, this in nothing. It's a simple iterative improvement.", Hroðulf concluded, albeit with a tone free of humility, his smile from before appeared again as he considered if he ought to say what he was thinking. His urge to tell won over his prudence to keep shut: 

"But I will tell you what is: sentiment. There are so many of these silly believes that by themselves becomes the agents of its own truth. For us, as individuals, spending less than we earn is the only viable strategy for long-term wealth; hence, it's thus assumed, that the same strategy ought to be applied at the aggregate level of total spending, and spending regarding the public. Naught could be further from the truth! What's true in the singular, might be the complete opposite in the aggregate.\\

Consider for example: if each animal saves more, then each business would receive less and subsequently have less to pay out in wages or earnings, which in turn would only decrease what each animal have to spend in the first place. Good individual decisions could thus cause a vicious spiral of less and less spending rendering a huge dent in our economy. Frugality would make us all the poorer! 

Savings bear no meaning in its aggregate, if you really think about it. If someone reduces his savings by way of consumption, it will become but the savings of someone else. Aggregate savings must always equal the total supply of money.\\

And, that leaves us with sentiment. It has been shown that animals saves less if they believe in a strong economy; and more for a weak one. How backwards is it that our economy is at the whim of fools' belief! What if I told you, there might be other ways for us to control demand?"\\

When Cinnamon responded to Hroðulf's conjectures, Kraerion had long stopped to listen. "The need to know the Time \textit{is} the problem.", he muttered, "No one makes plans to smell the hillside flowers, nor watch the glimmer of the morning dew; when the day's activities becomes shackled to Time, the wonders of the world does not cease to exist, but becomes imperceptible with no Time available to stumble upon them. The solution to Time, is indeed no solution at all.". 

Instead, he chipmunk admired the woodpeckers hard work. He was hit by sudden realization: weren't there too many a woodpecker for just one tree; and the roebucks were hauling dirt, but there was no sign of digging---whence came the dirt? But, what did Kraerion know about the process of industry; a subject he had so eagerly ignored to understand.\\

\renewcommand*{\thepage}{\footnotesize \mayadigit{42}}

When great ideas are discovered---those seemingly obvious in hindsight---they appear imperishable. There's no way to put the genie back in the bottle. They propagate from mind to mind---like a virus---only by the perceived virtue of the idea itself, until it's safely kept in the collective mind of society. Yet great ideas do perish; they slowly change, transforms, and morphs until it's something completely new and nothing of its original beauty remains.

We humans of today have learned to tell time in base 60, because long ago, when the ancient Babylonians created their first sundial they had in turn inherited a sexagesimal system from the even more ancient Sumer. Much of the meaning, its innate substance, of their system of mathematics have long been forgotten---or fallen out of relevancy, or transposed to what we consider math today---yet here today the arbitrary number 60, written in decimal, remains inert. Now is but the cumulative residue of legacy; the fabric of time woven with fragments of lost meaning.

Or is anything arbitrary---60 is a fortunate number for mathematics---maybe the Babylonians discovered how to measure time \textit{because} of their inherited numbers. Their discovery but the product of the trajectory of their past---propelling them to take the next progressive step of human understanding: could Newton, or Leibniz have developed calculus, if they were burden to do their long division with Roman numerals?

\renewcommand*{\thepage}{\footnotesize \arabic{page}}

The survival of ideas have by some been associated with Darwin's Natural Selection: knowledge---that is useful to its carrier(or not so useful, predatory even, as with the spreading of a cult)---is remembered and passed on. Information that is easily shared, like an infectious joke, or a compelling fireside tale, survives the erosion of time. And just like in biology, numbers like 60, like our useless appendix, still remains inert---a relic of the past without function but for the telling of time. The phenomena might best be exemplified with language, they slowly change, become dialects, merge and borrow words, some die and others spread to conquer the earth! But as with any stochastic process, they appear---insofar as we are capable to discern---as noise; and only when we zoom out, over the vast frames that is our history, we are able to see the undeniable signal.

Whatever our history has to teach, when Hroðulf saw a shadow and realized what it meant---finding a deeper truth. Did he consider the future crows, who would grow up burdened to learn his clock---that the arbitrary numbers of stones he chose for his sundial would be the last piece to remain from his existence on earth?
Did he consider the implications of his societal economic experiments, what society it would create, and that it, once set loose, would be unstoppable and irreversible when reaching motion?\\

By the time Cinnamon finally exhausted her curiousness, the sun was about to go down. The two chipmunks began their journey back to Kraerion's hollow. They took another way back, as to see other parts of the forest than on their way there. They wandered once again in silence, but this time it was different and free from tension. It was their silence, the one they had shared for so many years; a silence to grow old in. Both felt it, yet they likewise knew that their paths would soon diverge.

"You know, it's not that far between your borrow and my hollow...", Kraerion began cautiously.

"I know.", Cinnamon said slowly.

"You think we could give it a try?"

"I don't know, maybe...", she almost whispered, but then she nodded, "sure.". They didn't exchange any other word the short distance back to the hollow, they didn't need to. Kraerion thought back to Hroðulf's line, 'our sun is eternal bound', entertained the idea, and smiled a solemn smile at his own folly.
