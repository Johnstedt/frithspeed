
\section{Tier one, must be noted}

The story should follow a red line, even if nothing is understood from the two higher dimensions.


\section{Tier two, may be noted}

Missing to draw these connections should not interfer with the cohesiveness of the store,
but would propably diminish the read experience.

The story aims to depict the U.S. economy during the first half of the 20th century.
%Two world wars and the great recession was skillfully used 
%to gain support for increased government spending. Since such an increasing taxation wouldn't be feasable
%a shadow tax in form of currency debasedment was implemented.
%Their dollar, tied to gold, was debased multiple times and then completely decoupled from gold - initiating an eternal
%increase in money supply and inflation.


* 

Chapter 4 begins with a quote from Franklin D. Roosevelt's inagural speech.
Roosevelt used the great recession as a justification to implement his government
programs and unprecedented increase in government spending, and the Eaglewing
speech in chapter 4 is basically a rewriting of his inagural speech. 


- In Chapter 6, the ban of gold is Roosevelt's Executive Order _ and the subsequent 
rebasement of gold.
- 

* Vlahir oddly resembles John Maynard Keynes.

* Veritas, Aequitas and Pietas are all roman deities. 
- Veritas, or truth is both used to express the truth and his death is thus both literal and figurative.
- Aequitas, or equality, rots away with the death of veritas. While equality is positive and something to strive 
for, it's but a product of other factors. 
- Pietas, or duty. Duty is a dubble-edged sword, without truth nor equality, it only adds pain.
- The quote on equality and freedom, is a quote by Milton Friedman. 


* Aesop is the name fabelist who was the first to write stories with a moral(Swedish 'sensmoral' is a better word since it clerly differes from morality). He "made use of humble incidents to teach great truths" - Apollonius. Feels fitting that the only caracter with morals is named
after Aesop.
* Snowball is a character from Animal Farm, as this book has so many other similarities to Animal Farm, the otherwise ambiguis name is obviously a reference.
- Stoats is similar to the dogs in Animal Farm.
* Zrefrafa is a reference to the rabbits in Watership Down and is similar to Efrafa.

\section{Tier three, impossible to note}

Where the reference, or connection are so vague that it's likely to be a coincidence.
But for anyone noting.

* The man with tooth ake is a man in Dostoevsky's Notes from underground.

* The Cinnamon and Kraerion story is somewhat based on Kunderas writing. I can't 
pretend that I'm able to write of a dynamic relationship so I had to borrow.
- The eternal return segment is Kunderas idea of Eternal Returns.
- The weight attributed to Kraerions feelings is intentional, being "light" in
the tree and heavy with Cinnamon. This is the fantastic opposite dilemma that
Kundera likes to explore, "The unbearable lightness of being" just being the 
top of the stack.
- To be honest, I just tried to make the loss feel more significant.