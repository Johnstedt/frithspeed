
\chapter{Wadborough forest}
\vspace{-1.3cm}
\begin{localsize}{10}
	\begin{quote}
		"Man selects only for his own good; Nature only for that of the being which she tends."
		\begin{flushright}- Charles Darwin\end{flushright}
	\end{quote} 
\end{localsize}
\vspace{1cm}

The Wadborough forest is a peculiar piece of land. It's the only forest for miles and stands out in the messy patchwork of farms and meadows surrounding the village of Wadborough. When the popluation boomed with the industrial revolution; the forests were cut and the lakes were drained to feed the increased mouths. But not Wadborough forest; the Duke of Worcestershire, to be popular with his peers, began importing pheasants and kept the forest as a hunting grounds for his more distinguished subjects.\\
 
Pheasants are clumsy birds, used to roam the Asian steps undisturbed from predators, and found it difficulty to survive in the forest of Wadborough. To protect their precious game; the villagers drove, over the years, all major predators out of the region. The red fox was first to go, as they seemed to hunt the pheasants out of sport - killing more than they could eat. The weasle family: the stoats, the wolverines, and the mink; followed shortly thereafter. The badgers quickly learned to avoid the clumsy bird and were left alone after reaching low numbers. Birds of prey were also kept as they mostly ate rodents which consequently were booming with the removal of their natural predators. The area even became famous among ornithologists for the many prosperous goshawks and species of owls.\\

Times changed; people moved to cities and soon the rural arts were lost to a majority of citizens. Hunting, especially when performed for sport as with pheasants, became viewed as cruel and barbaric. Activist groups formed and rallied - the bill of wildlife preservation act was past in the county of Worcestershire which holds the legislative power in Wadborough. It contained many paragraphs, one of which dealt a devestating blow to Wadborough forest.

\begin{localsize}{10}
\begin{quote}
'A person shall not hunt game birds including but not limited to the Common Pheasant, Grouse, Goose, Turkey, Duck or Pigeon by means of firearm, or any form of projectile, unless bread in captivity under permission from a state licensed breeder. A person guilty of offense under this Act shall be liable on summary conviction to a fine not exceeding 5.000 pound sterling.'
\begin{flushright}- §68, Bill of Wildlife Preservation Act\end{flushright}
\end{quote} 
\end{localsize}

"What does city folk know of hunting?" was heard in many a pub, market, and home after the bills passing; the villagers did not abide by the new law. After a change in mayor and tens of hefty fines, they eventually followed suit.\\

And so it came to be that the forest of Wadborough was free of both man and predator — undisturbed by the natural checks and balances that keeps the order of things. Even the locks of gold couldn't compare to the beatiful era that would follow.