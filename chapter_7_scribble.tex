Use Veritas (the internal intellectual) or Áísōpos (the external truth-teller) to define and immediately refute the terms, turning the clarification into an act of resistance.

Weaving through Veritas: If Veritas were to write her findings, she could define the term before attacking the concept:

"They speak of Fungibility—that all leaves are equal. Yet they do not tell you that they control the flow, making that equality an illusion, easily dissolved by the Crow's own decree."

Weaving through Áísōpos (Simplified Refusal): Áísōpos can reject the jargon entirely, which simultaneously defines it for the reader:

Áísōpos snorted. "Spare me your talk of traceability. It is not 'traceability' but spying! It is the removal of the right to keep one's transactions private. Call it what it is, Crow, not by some lofty word."

By embedding the definitions in the characters' dialogue and actions, you clarify the jargon without having the narrator interrupt the flow or break the tone with a glossary entry.


\textbf{The Animal Death Convention:} The line "(Animals seldom die in burrows, but seek a quite place to die)" is good information, but placing it in parentheses breaks the narrative flow.

\begin{itemize}
\item \textbf{\textit{Action:}} If this detail is important, it should be integrated into the narration, perhaps as Kraerion realizes the horror of the scene: "Even in death, she had defied Nature; animals seldom died in burrows, but sought a quiet, open place. Veritas had instead chosen this confined space, sealing herself in with the very truth she preached."


\end{itemize}
