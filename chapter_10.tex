\chapter{Duplicity}

\vspace{-1.3cm}
\begin{localsize}{10}
	\begin{quote}
		"Concentrated power is not rendered harmless by the good intentions of those who create it."
		\begin{flushright}- Ayn Rand \end{flushright}
	\end{quote} 
\end{localsize}
\vspace{1cm}

% Averruncus
The old oak tree stood eerily grey in the otherwise lush green glade, its stiff leafless branches swayed naked in the soft cold wind. The trunk, now full of crow apartments, had been too damaged to support the high branches with nutrients and water. Rot had started to show, as larvaes of all kinds feasted on its roots. The glade was still if not for three roebucks who dragged a huge rectangular piece of limestone slowly towards the oak.

"Where do you want this?", one of them said as they caught their breath.

"In the opposite end of the glade", Eaglewing answered from a bare oak branch, "and raise it on its short side."

"And how do you expect us to do that?", the roebuck responded irritably, but instead of waiting for a reply he burried his head and the roebucks continued to slowly drag the limestone across, whilst Eaglewing watched amused with a sly grin covering his face.  
	
Eaglewing had not always been Eaglewing. He had been named Óttarr after his father's great grandfather. Even as a hatchling his brown left wing stuck out like a sore thumb and he had earned the name Mudwing by some older fletchling bullies. The nickname stuck, as ill intented names tend to do, and when Eaglewing had grown past his third winter he had had enough. He  complained to his father about his petty nickname. His father, then the chief crow, had answered, "Nicknames, especially those dispensed with malice, are quicksand — the more you squirm, the deeper it sinks. The solution, I recon, is no different from that of getting any policy approved, it is only an affair of shaping public opinion. You mustn't directly oppose your peers, you must appear to agree; float along, like a log in a river, and once you find yourself in the front, slowly lean and bend the river until the current is yours. Once you steer rivers, there's no landscape you can't mold and no opposition you can't erode"

Eaglewing, eager to gain his fathers approval and rid his foul name, began to hatch a plan. The first week he only stopped taking offence when someone used it. The second week he even pretendet to enjoy it. By the time the third week came, he preached to anyone who cared to listened about how wonderful the maroon colour of mud was and how it shone in limelight. He acted deliberatly pretentious, as a madman high on himself, setting bait, giving everyone reason to knock him down a notch. Gradualy moving the needle at a speed below perception as his father had tauth him

He made no one expect close friend, Brightbeak, aware of his plan as he needed his help in his scheme. When the day he had been waiting for finally came, when his plans would yield dividend, he and Brightbeak walking along with the other fletchlings including the once who had given him his name.

"Eagles are the rats of heaven", said Mudwing unprovocted.

"No, they are not", snapped \_, the largest of them. "They are majestic, I've seen one myself"

"I doubt that", answered Mudwing, "If thou hadst, thou wouldst know how ugly they are with their brown wings."

"But Mudwing", said Brightbeak, "Your wing is brown."

"No, my wing is maroon", 

"No it does, it's the same colour as the eagle's I saw", said \_\_ slyly, "You have an eagle's wing."

"How dare you compare me with one of those flying rats", cried Mudwing

"I think we all know what this calls for", said another of the larger crows which made them all laught. \_\_ was the first to catch his breath so he said,  

"Surely we do, do you Eaglewing?"

***

When Mudwing became Eaglewing, Ongenþeow took a larger interest in his crippled son. He began grooming him for office. He took him on many a trip to the surrounding towns and sometimes even larger cities. One time he brought him to a statue of one Lord Nelson and Ongenþeow croaked, 

"See here Óttarr, this man did many a winter ago, yet here his likeness stands erect---imprinted in the collective memory of man forevermore."

He also explained---when they sat on a branch in a public park in a larger city---that most humans here did not plow the fields for their livelihood like in Wadborough---instead they, 'play with paper and numbers', and how others bring it to them. "How can a man of the fields agree to such an arrangement?", Ongenþeow asked his son, "It's certainly require a moment of contemplation."

The crow brought his son to his favorite stone, and on the stone weird signs were chiseled:\\


% My used symbols
\newcommand\SUMb{\symbol{"122F3}}
\newcommand\MA{\symbol{"12220}}
\newcommand\A{\symbol{"12000}}
\newcommand\WI{\symbol{"1227F}}
\newcommand\LUM{\symbol{"1221D}}
\newcommand\I{\symbol{"1213F}}
\newcommand\IN{\symbol{"12154}}
\newcommand\DUMU{\symbol{"12309}}
\newcommand\LIM{\symbol{"12146}}
\newcommand\UH{\symbol{"12314}}

\newcommand\TAB{\symbol{"122F0}}
\newcommand\PIRIG{\symbol{"1228A}} % BAD
\newcommand\IT{\symbol{"12009}}
\newcommand\SU{\symbol{"122D7}}
\newcommand\Ub{\symbol{"12311}}

\newcommand\HA{\symbol{"12129}}
\newcommand\AP{\symbol{"1200A}}
\newcommand\PA{\symbol{"1227A}}
\newcommand\DU{\symbol{"1207A}}

\newcommand\GIR{\symbol{"1210A}}
\newcommand\PAD{\symbol{"1227B}}
\newcommand\IS{\symbol{"12156}}
\newcommand\TE{\symbol{"122FC}}
\newcommand\BE{\symbol{"12049}}

\newcommand\ER{\symbol{"12155}}
\newcommand\SE{\symbol{"122BA}}
\newcommand\EB{\symbol{"12141}} 
\newcommand\RU{\symbol{"12292}}


{\setmainfont{Akkadian}%Noto Sans Cuneiform}
%\rotatebox{270}{
\centering{
		\Large 
		
		\SUMb\MA \A\WI\LUM 
		
		\I\IN \DUMU \A\WI\LIM
		
		\UH\TAB\PIRIG\IT
		
		\I    \IN      \SU
		
		\Ub\HA\AP\PA\DU \\[0.5cm]
		
		
		
		\SUMb\MA \GIR\PAD\DU
		
		\A\WI\LIM
		
		\IS\TE\BE\ER
		
		\GIR\PAD\DU\SU
		
	    \I\SE\EB\BE\RU
	    
	}
		
%	}
}


  
"What do they mean?", Óttarr asked like any curious child would.

"I don't know," his father answered patiently, "but it was apparently written by a king very long ago, in a land very far away, in a language no one knows what it means. But the name of the king is known however: it's Hammurabi." 

