\chapter{Denial}

\vspace{-1.3cm}
\begin{localsize}{10}
  \begin{quote}
    “We've tied happiness to guilt. And we've got mankind by the throat.”
    \begin{flushright}- Ayn Rand \end{flushright}
  \end{quote} 
\end{localsize}
\vspace{1cm}



The beautiful long days of summer had past without Kraerion spending more than a few moments above ground. He spent most of his awaken hours working, carrying goods in his cheeks whilst orchestrating his instrument of chipmunks on what and which should go where and when. When he for a moment peeked out of one of the entrences one October morning, he was suprised to see the trees already flaming bright, the canopies covered in the orange colours of autum. "Where did the summer go?" he wondered solemnly as he realized he hadn't had the time to reflect on times passing. His thoughs quickly turned to anguish as he realized how close it was to winter and how much work was due by then. Although he felt like he pushed himself to the edge of sanity, he was kept humble be his uncle. Pietas worked close to 18 hour long days, sleeping in the Aorta as to always be on call if he was needed. He never complained, nor did he uttered any negative comment at all. Kraerion was very greatful for that, all chipmunks respected and when they eventually needed to increase the working hours most had gone along with it since Pietas stood fully behind the decision.


%He had returned home, but not like he'd imagined: After he'd found himself and could know the place for the first time.


%He'd returned home, but not like the heros in the stories: who returns to arrive where he started and know the place for the first time. Because this is not one of those stories. This story is not about heros.

%One may be apt to "to arrive where we started And know the place for the first time.". This is not one of those stories.

%In the stories, the hero arrives where he started and know the place for the first time. But this story is not one of those—this story has no hero.

Kraerion had arrived where he started, but unlike the T.S. Eliot poem, he didn't know the place for the first time: it was the same damp dirt he'd grown to loathe. This is not your archetypical hero's journey; in fact, this story has no hero.

The sun had long gone down when Kraerion finally made it home. Their narrow living space was more gutter than burrow. The damp earthen walls leaning close, swallowing the burrow as the black saharan night swallows the glimmer of a lone dragonfly. Cinnamon had birthed three kittens earlier that spring, and as Kraerion entered, he paused for a momement at the entrance looking in. Their room was worse than his former birch appartment in every measureable way, yet it had a warmth only a family could bring, their shared memories wearing the walls, filling them with life and giving them a heavy invaluable quality and making it a home. As Kraering stood there, watching his kittens, he noticed how much they grown. "Trees are not the only thing I miss grow", he reflected solemnly.

"Are you not coming in?", Cinnamon asked, interupting his thoughts. 

A silence to build a home and be called lucky for it.

A sort of silence in which to build a house, move in, and call a home. 


 
Kraerion had sold his hollow in the birch tree. A crow had offered him trice what he had spent to acquire it, yet the amount, which he promptly donated to the hungry, had been but a drop in the bucket in a vast see of hungry wavs. 
His hollow in the tree had been everything, his identity and way of life. It had been invaluable. In that value is subjective, to each his own, but in the end it had a face value. A face value that a crow, without much of a care, were willing to spend. "A nice place to spend summer weekends", he had said to Kraerion, "those were the sun only hides for an hour or two". Valued differently and equally, a paradox only a shift in denomination may cause where the past discounted and posterity borrowed. Where the frugal is punished at the hand of the lavish. Where spending is the only alternative left, save for spending some more.



Some create history, some write it; but most of us are simply there to be affected by it.

When some crow decades from now sit down to write the histories of this time period, Kraerion will not be mentioned. When a rabbit with a skill of word and a unusual empatic heart writes the 'Animals history of the Forest', Kraerion will not even be a statistic, dreams and hope cannot be accounted for in the world of numbers. Eaglewing and Hrolf will surely be included, their statues remain erect. As history has taught us, not even the Great Famine in China, nor the Holdomore in today's Ukraine, could raze the statues of the people responsible to the ground. Alas, they might have had some humanitarian qualities, they were nice enought to remained their fellow subject, after all, that: to eat your own children is a barbarian act.

Kraerion is not a remark in passing of history. And we only feel sympathy with him because his story is told. Dreams forgone is seldom accounted for in the tables of the statistician. Neither is the privilege to pursue happiness. The fortunes and tragedies of regular people are not easily aggregated;  yet, they many of them are the product---in the millions---of some intellectual with ideas about the economy. Macroeconomics is a boring pursuit, yet it is the mother of all other pursuits.




