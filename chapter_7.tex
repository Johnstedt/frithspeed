\chapter{Discrepency}

\vspace{-1.3cm}
\begin{localsize}{10}
	\begin{quote}
		%“In man's struggle against the world, bet on the world.”
    %\begin{flushright}- Franz Kafka \end{flushright}
    “Until they become conscious they will never rebel, and until after they have rebelled they cannot become conscious.”
		\begin{flushright}- George Orwell \end{flushright}
	\end{quote} 
\end{localsize}
\vspace{1cm}

The winter came early and hard. Most of the song birds had long migrated south, but the pidgeons made up for it with their incessant cooing. The forest bed was covered in about five inches of snow and most animals were in their burrows and nests trying to preserve their body heat. Kraerion had insulated his hollow, the inner walls meticulously covered with moss and lichen. Redrill had carved a rounded wooden piece, which could be rolled in front of the entrance to keep the warmth from escaping.

In the mornings though, he pushed the piece aside. He preferred the view of the glade over warmth. He had watched the morning dew slowly been turned to frost. The years before he had been underground and not fully realized that they were but two sides of the same metal, denominated by temperature. His mind wandered back to the events in the glade. The seizure of Aesop, the death of the brave stoat and the forbidding of screw-nuts. No one had seen Aesop since he fled, yet everyone seemed to carry their own rumor where he was or what he was doing. The hedgehog Spinestack had sworn he'd seen the stoats catch him, and that he most likely was already buried as to away trial. The pigeon Greyhead conjectured that Aesop had escaped the forest and sought rebels in the surrounding meadows and marshes. No one knew for certain, yet it didn't keep anyone from talking about it. Kraerion didn't know what to make of it. Aesop was peculiar, no one denied that, but a theif? No, he could not believe it.

He's thoughs were abruptly disturbed by Cinnamon who suddenly appeared in the glade, as she had done that late summer morning. This time she didn't stop at the foot of the birch, rather she raced up trunk with a fury not to her character. When she reached the hollow she was completely out of breath, grasping for air - Kraerion did not need to hear her speak, he saw in her pearl black eyes that she bore grave news. She knew there was no point in delaying the inevitable, so when she caught breath she uttered solemnly 

"Shrub is dead", with a slight shiver in her voice. Her face carried grief, but turned empathic in an instance once she saw Kraerions reaction, "I think you better come."

They ran side by side back to the tunnels and into Shrubs burrow. They ran in a, to them, new sort of silence. A heavy silence. A Silence which could not bear the weight of words. Nothing carry intent as silence does, the implicit understanding of words not uttered. 

They entered the Shrub's burrow and joined the small crowd of chipmunks surrounding his body. Aequitas was leaning over him with his forepaw pushing lightly on his chest, not hiding his grief. Next to his father stood his uncle, Pietas, who bore a refrained expression over his face. Around them stood a mixture of kin and friends. Veritas had been a difficult fellow to enfriend, but no one denined his ability to keep accounts and his integral part in their shared vocation. When Kraerion saw his brothers, his mind ignited, filling him with memories. Their  quarrels, their laughter and the childhood they had shared. Memories that now had grown heavy with grief.

Kraerion and Cinnamon pushed their way forward through the crowd to join Aequitas and Pietas next to the corpse. Aequitas, still with his paw on the chest, turned to face the two arrivals. His face was full of grief. 

"What.., what happened?", Kraerion managed to say, breaking the long silence.
"I.., He..", Aequitas began, but his voice gave way. He took a deep breath, "I.. I think he starved himself to death.. He had a strong conviction that we'd not collected enough food to survive the winter. That we ought to ration. I.. I know he was eating less.., but to so far as to..", He struggled to finish the sentence, as if stating the fact would remove all hope that this was just a bad dream. Instead he said, "He's not wrong you know.., the food supply is running short. Veritas is not the only one who'll starve to death before this winter is over."

"What!?", Kraerion said, "What do you mean? Didn't you tell the whole forest that our cellars are filled to the brim?"

"Gnath", Aequitas scoffed, "Eaglewing came to us the day before and told us not to tell about the deficit. 'Panic would ensue', he claimed and I.. I believed him. But Veritas didn't, he simply would not have it. Eaglewind and myself agreed that it was for the best to leave Vertias out of the meeting. I.. I was naive, I thought we ougth to have had enough time to quietly fix the problem before chaos ever surfaced. But it doesn't seem like it anymore..."

Kraerion was stupefied and when he didn't respond, Cinnamon took the tone. "But how? How could this be kept a secret?", she asked, "Don't we all keep track of how much food we've stored?"

"You would think that was the case", Aequitas snarled, "But most animals doesn't run in our tunnels, nor visit the our cellars. They simply put away bits and pieces throughout the year and hope it will be enough. And this year it's simply not enough."

Cinnamon paused for a moment before responing. "If what you're saying is true", she said, "We must inform everyone."

"It's no use", Aequitas cried verily with a deep and troubled voice, "The forest is cover in a thick layer of snow. It's no use."

The cry ended the conversation. The four chipmunks and their fallen friend remaind silent for a very long while, each grapsing the dire consequences and fighting their imidiate grief. The days passed but Aequitas never left Veritas bedside. Any will to work or deal with the chipmunks daily business were gone with Veritas. Next to him remained, broken by grief, until eventually he fell dead himself. 

A great man once claimed that a society that put equality before freedom will get neither. If truth is a prerequisite to freedom, what would that imply for equality? What happens to agency when the Great Truths aren't allowed to propagate? Equality is ever so important, yet it's not a factor - it's a product. It's not the reason, it's the result. The end result of a well governed 

Like a great eastern power has tauth ous with their devotion to Orwell. Take away the thruth and you take away options, without option we behave predictable to a tee, droning on the only way we know how. Truth is a pre-requesit to freedom and freedom is a pre-requesit to equality, 

A 'free' choice is not free when choices are censored and can't explored. A limited agency, where all the given options are controlled and managed. Although no explicit law or rule forbids your freedom of choice, the implicit imprisonment of truths denied. 