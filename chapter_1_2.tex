
\chapter{Deordination}
\vspace{-1.3cm}
\begin{localsize}{10}
	\begin{quote}
		"Man selects only for his own good; Nature only for that of the being which she tends."
		\begin{flushright}- Charles Darwin\end{flushright}
	\end{quote} 
\end{localsize}
\vspace{1cm}

The Wadborough forest is a peculiar collection of trees; it's shaped by an improbable set of circumstances, setting the preconditions of this unlikely tale you're about to read. % shaped by an equally peculiar set of circumstance; setting the unique condition of this unlikely tale.

The forest has stood where it now stands for millennia, but surrounded by other trees and part of a much larger forest: it did not crave attention, nor require distinction by name. With the ever increasing sophistication---and population grow- th---of the main British isle, most trees were cut down for timber, and the woodlands were slowly turned to grazed meadows and tiled farmlands. The trees near the little village of Wadborough would most certainly meet the same fate, if it wasn't for a growing posh ritual of high British society. The Duke of Worcestershire---lest he lose pace with his peers---began importing pheasants to be hunted by his guests and his more distinguished subjects. He thus proclaimed that the trees near the Wadborough village would never be cut down, and ever solely be allocated to the hunting of game. These trees became the Wadborough forest---standing tall and alone in the wild ocean of fields.

Pheasants are clumsy birds---used to roam the Asian steps undisturbed by predators---and found it difficult to survive in the forest. To protect their precious game: the Duke's men drove, over the years, all major predators out of the region. The red fox was first to go(to the chickens' rejoice) as they seemed almost offended by the ease of hunting pheasants. The weasel family: the stoat, the wolverine, and the mink; followed shortly thereafter. The badgers were also hunted, but they soon understood that the pheasants were off limit and began to ignore the foreign bird; thus, a few badgers remained alive. As with all human interventions: they're seldom without unforeseen consequences; the rodent population boomed; which, incidentally attracted more birds of prey to the region. These birds---mostly consisting of various hawks and species of owls---did not danger the life of pheasants; however, they possessed another threat entirely. The ornithologists of the time had a fierce reputation of enacting conservation legislation; the rumor of the birds broad prevalence reached the group and soon they took interest in the forest. The Duke thus had a choice: spare the birds and risk the ornithologists' influence, or kill them and risk the outbreak of rodents; he chose the latter. The rodents did explode in numbers, but they did not carry disease with them, nor overrun the neighbouring fields and the Wadborough village. As a matter of fact---the rodents did not at all conform to the behaviours so often attributed to them. No one could explain why.

Eventually the time of serfdom met its end; and the society slowly morphed into something approaching the appearance of a representative democracy. Ownership of land transferred from lords and ladies to the establish municipalities, and so too did the ownership of the Wadborough forest. Although hunting remains a tradition of the well-off, its clientele slowly shifted to in time mostly consist of ordinary rural villagers.   
Innovations in agriculture and growing manufacturing greatly shifted the populace from rural to urban. The growing cities lost touch with the rural arts. Hunting---especially when performed as sport as it is with game---became viewed as cruel and barbaric. The urban population greatly outnumbered its rural counterpart in voting power; activists rallied and tallied support---the Bill of The Wildlife Preservation Act soon passed through the county of Worcestershire; which, conveniently holds the legislative power in Wadborough. The Act contained many a paragraph, but only one concerned the Hunters of Wadborough. 

\begin{localsize}{10}
	\begin{quote}
		'A person shall not hunt game birds including but not limited to the Common Pheasant, Grouse, Goose, Turkey, Duck or Pigeon by means of firearm, or any form of projectile, unless bread in captivity under permission from a state licensed breeder(§95b). A person guilty of offense under this Act shall be liable on summary conviction to a fine not exceeding 5.000 pound sterling.'
		\begin{flushright}- §68, The Wildlife Preservation Act\end{flushright}
	\end{quote} 
\end{localsize}
 
The Act angered the villagers; "what does city folk know of hunting!?" could be heard at the local drinking holes. Wild conspiracies was liberally spread along with wild guesses on the probabilities of actually getting caught. To the villagers dismay, the city clerks had foreseen their unwillingness to abide by the new law and were ruthlessly prepared to hand out many a salted fine during the first year of the bill's passing. 

And so it came to be that the forest of Wadborough was free of both predator and man alike---undisturbed by the natural checks and balances that keeps the order of things. Even the locks of gold couldn't compare to the beautiful era that would follow.


